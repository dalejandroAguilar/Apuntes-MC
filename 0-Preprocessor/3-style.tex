%%Para centrar columnas de ancho especifico%%
\newcolumntype{L}[1]{>{\raggedright\let\newline\\\arraybackslash\hspace{0pt}}m{#1}}
\newcolumntype{C}[1]{>{\centering\let\newline\\\arraybackslash\hspace{0pt}}m{#1}}
\newcolumntype{R}[1]{>{\raggedleft\let\newline\\\arraybackslash\hspace{0pt}}m{#1}}

%%tamaño de márgenes en hoja%%
\headsep = 14pt
\topmargin = -20pt
\headheight = 14pt
\footskip = 30pt
\textheight = 690pt
\textwidth = 480pt
\marginparwidth = 10pt
\oddsidemargin = 0pt
\marginparpush = 0pt

%%opciones de encabezado fancy%%
\lhead{}
\chead{}
\rhead{}
\lfoot{Daniel Aguilar}
% \cfoot{\thepage}
\fancyfoot[LO,RE]{\thepage} 
\renewcommand{\headrulewidth}{0.5pt}
\renewcommand{\footrulewidth}{0.5pt}
\fancyheadoffset{0cm} %clave para que coincida la line header en longitud con el ancho de texto
\fancyfootoffset{0cm}
\pagestyle{fancy}

\makeatletter
\def\thickhrulefill{\leavevmode \leaders \hrule height 1.2ex \hfill \kern \z@}

\def\@makechapterhead#1{
  \vspace*{10\p@}%
  {\parindent \z@ \centering \reset@font
        \thickhrulefill\quad 
        \scshape\bfseries\textit{\@chapapp{}  \thechapter} 
        \quad \thickhrulefill
        \par\nobreak
        \vspace*{10\p@}%
        \interlinepenalty\@M
        \hrule
        \vspace*{10\p@}%
        \begin{table}[H]
    		\begin{subtable}{.15\textwidth}
    			\centering
    			\color{rojoASTM} \fontsize{50} $\bra{\Psi}$ 
    		\end{subtable}
    		\begin{subtable}{.69\textwidth}
    		\centering
    			\Huge \bfseries #1 
    		\end{subtable}
    		\begin{subtable}{.15\textwidth}
    		\centering
    		\color{rojoASTM}  \fontsize{50} $\ket{\Psi}$ 
    		\end{subtable}
    	\end{table}
    	\par\nobreak
                \par
        \vspace*{10\p@}%
        \hrule
        \vskip 100\p@
  }}