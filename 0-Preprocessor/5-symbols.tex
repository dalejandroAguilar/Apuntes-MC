%!TEX root = ../main.tex
\opensymdef
\newsym[atomic number]{sZ}{Z}
\newsym[number of neutrons]{sN}{N}
\newsym[basic charge]{se}{\ensuremath{e}}
\newsym[]{symr}{\ensuremath{\rho}}
\newsym[Conductancia]{symC}{G}
\newsym[Resistencia]{symR}{R}
\newsym[Fuerza]{symF}{\textbf{F}}
\newsym[Campo Eléctrico]{symE}{\textbf{E}}
\newsym[Campo Magnético]{symB}{\textbf{B}}
\newsym[Schrödinger]{perSch}{\text{Schrödinger}}

\newsym[Hilbert Space]{hilb}{\ensuremath{\mathcal{H}}}
\newsym[Hilbert Space]{pro}{\ensuremath{\mathcal{P}}}

\newsym[Hilbert Space]{brapsi}{\ensuremath{\bra{\psi}}}
\newsym[Hilbert Space]{braphi}{\ensuremath{\bra{\phi}}}
\newsym[Hilbert Space]{brachi}{\ensuremath{\bra{\chi}}}

\newsym[Hilbert Space]{ketpsi}{\ensuremath{\ket{\psi}}}
\newsym[Hilbert Space]{ketphi}{\ensuremath{\ket{\phi}}}
\newsym[Hilbert Space]{ketchi}{\ensuremath{\ket{\chi}}}

\newsym[Hilbert Space]{braketab}{\ensuremath{\bra{\psi}\ket{\phi}}}
\newsym[Hilbert Space]{braketba}{\ensuremath{\bra{\phi}\ket{\psi}}}
\newsym[Hilbert Space]{braketac}{\ensuremath{\bra{\phi}\ket{\chi}}}
\newsym[Hilbert Space]{braketca}{\ensuremath{\bra{\chi}\ket{\psi}}}
% Usar \ensuremath en símbolos matemáticos
\newsym[Definition]{symlam}{\ensuremath{\lambda}}
\closesymdef

%\listofsymbols %Mostrar lista de símbolos