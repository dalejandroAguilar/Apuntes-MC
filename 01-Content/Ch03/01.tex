\section{Operadores de evolución temporal}
\[
\ket{\psi}\then\vu{A}\ket{\psi}=\ket{\psi}
\]

Ahora, queremos ver cómo los estados evolucionan.

\begin{definition}[Operador de evolución $\vu{U}$]
\begin{align*}
    \vu{U}:\hilb&\then\hilb \\
    \ket{\psi(t)}&\then\vu{U}\ket{\psi(t)}=\ket{\psi(t+\Delta t)}\\
\end{align*}

\end{definition}

En la base generadora $\qty{\ket{e_i}}$ se tiene

\begin{equation}
    \ket{\psi(t)}=\sum_i\alpha_i(t)\ket{e_i}
    \label{eq:e31}
\end{equation}

\begin{equation}
\centering
    \ket{\psi(t+\Delta t)}=\sum_i\alpha_i(t+\Delta t)\ket{e_i}
    \label{eq:e32}
\end{equation}

\begin{align*}
    \alpha_i(t+\Delta t)&=\bra{e_i}\ket{\psi(t+\Delta t)}\\
    &=\mel{e_i}{\vu{U}}{\psi(t)}
\end{align*}

Usando \autoref{eq:e31}

\begin{align}
    \alpha_i(t+\Delta t)&=\sum_j \mel{e_i}{\vu{U}}{\psi(t)}\alpha_j\\
    &=\sum_j U_{ij}\alpha_j(t)
    \label{eq:e33}
\end{align}

$\vu{U}$ en la base $\qty{\ket{e_i}}$:

\[
    U_{ij}=\mel{e_i}{\vu{U}}{e_j}
\]

¿Qué forma tiene $U_{ij}$?

En el momento inicial $\Delta t=0$

Con \autoref{eq:e33}
\[
    \alpha_i(t)=\sum_j U_{ij}\alpha_j(t)
\]

Entonces $U_{ij}=\delta_{ij}$

Ahora para un istante posterior $\Delta t\then 0$ y $\Delta\not=0$

\[
    U_{ij}=\delta_{ij}+\Delta t K_{ij}
\]

Tomo deliberadamente $K_{ij}=-\frac{\vb{i}}{\hbar}H_{ij}$.

\begin{equation}
\centering
   U_{ij}=\delta_{ij}-\frac{\vb{i}}{\hbar}\Delta H_{ij}
   \label{eq:e34}
\end{equation}

Remplazo en \autoref{eq:e33}

\begin{align*}
    \alpha_i(t+\Delta t)&=\sum_j \qty(\delta_{ij}-\frac{\vb{i}}{\hbar}\Delta t H_{ij})\alpha_j(t)\\
    &=\sum_j\delta_{ij}\alpha_j(t)-\frac{\vb{i}}{\hbar}\Delta t\sum_j H_{ij}\alpha_j(t)\\
    &=\alpha_i(t)-\frac{\vb{i}}{\hbar}\Delta t\sum_j H_{ij}\alpha_j(t)
\end{align*}



\begin{align*}
    \text{lim}_{\Delta t\then 0} \qty[\vb{i}\hbar\qty(\frac{\alpha_i(t+\Delta t)-\alpha_i(t)}{\Delta t})]&=\sum_j H_{ij}\alpha_j(t)  \\
    \vb{i}\hbar \dv{x} \alpha_i(t)&=
\end{align*}

\begin{equation}
    \vb{i}\hbar \dv{x} \alpha_i(t)=\sum_j H_{ij}\alpha_j(t) 
    \label{eq:coef_ev}
\end{equation}

Así se puede definir a \autoref{eq:coef_ev} como la ecuación de evolución temporal de los coeficientes de un estado.

Del mismo modo $\ket{\psi(t)}=\sum_i\alpha_i(t)\ket{e_i}$ se define como la ecuación de evolución temporal de los estados.

\begin{align*}
   \vb{i}\hbar \dv{x} \bra{e_i}\ket{\psi(t)}&=\sum_j\mel{e_i}{\vu{H}}{e_j}\bra{e_j}\ket{\psi(t)}\\
    &=\mel**{e_i}{\vu{H}}{\psi(t)}\qq{(Propiedad de clausura}\dyad{e_j}=I)\\
    \qq{(Sólo}\ket{\psi(t)}\qq{depende de t)} \mel**{e_i}{\vb{i}\hbar \dv{x}}{\psi(t)}&=
\end{align*}

\begin{equation}
\centering
   \vb{i}\hbar\dv{t}\ket{\psi(t)}=\vu{H}\ket{\psi(t)}\qq{Ecuación de \perSch ket}
   \label{eq:sch_ket}
\end{equation}

Considere que $\vu{H}$ es un operador hermítico por lo cual $\vu{H}=\vu{H}^*$, entonces

\begin{equation}
\centering
   -\vb{i}\hbar\dv{t}\bra{\psi(t)}=\vu{H}\bra{\psi(t)}\qq{Ecuación de \perSch bra}
   \label{eq:sch_bra}
\end{equation}

Con relación a \autoref{eq:sch_ket} y \autoref{eq:sch_bra} se toma a consideración 

\begin{enumerate}[a)]
    \item La ecuación de \perSch nos da la evolución temporal de los \textit{estados puros} del sistema.
    \item Es una ecuación lineal, por lo cual se satisface el principio de superposición.
    \item Durante la evolución del sistema la norma permanece constante.
    \item Es la ecuación cuántica no relativista.
    \item Describe partículas sin considerar el spín de la partícula.
\end{enumerate}
\begin{proof}
    Procedamos a demostrar el literal c).
    \begin{align*}
        \ket{\Psi(t)}\qq{Evoluciona con el tiempo}\\
        \dv{t}\norm{\psi(t)}^2&=\dv{t}(\bra{\psi(t)})\ket{\psi(t)}+\bra{\psi(t)}\dv{t}(\ket{\psi(t)})\\
        \dv{t}\braket{\psi(t)}&=\frac{1}{-\hbar}\mel{\psi(t)}{\vu{H}}{\psi(t)}+\frac{1}{-\hbar}\mel{\psi(t)}{\vu{H}}{\psi(t)}\\
        &=0\then\norm{\psi(t)}=\text{const}
    \end{align*}
\end{proof}



\begin{example}[]
    Se conoce de un sistema de partículas de spin 1/2 los valores medios de $\expval{S_x}\qc\expval{S_y}\qq{y}\expval{S_z}$. Con esta información determinar la matriz de densidad.
    
    \textit{Solución}
    
    La matriz de densidad tiene la forma
    \[
    \rho=\mqty(a&b\\c&d)
    \]
    Conocemos que las matrices para los spines son:
    
    \[
    S_x=\mqty(0&\hbar/2\\\hbar/2&0)\qc S_y=\mqty(0&-i\hbar/2\\i\hbar/2&0)\qc S_x=\mqty(\hbar/2&0\\0&\hbar/2)
    \]
    
    Se debe cumplir que
    \[
    \expval{S_x}_{\vu{\rho}}=Traza(\rho S_x)\qc\expval{S_y}_{\vu{\rho}}=Traza(\rho S_z)\qc\expval{S_z}_{\vu{\rho}}=Traza(\rho S_z)
    \]
\end{example}

\begin{example}[Evolución temporal de un estado mezclado]
    Sea un sistema físico $\vu{\rho}=\sum_i \ket{\psi_i}P_i\bra{\psi_i}$ Calcular $\dv{t}\vu{\rho}$.

    \textit{Solución}
    
    \begin{align*}
        \dv{t}\vu{\rho}&= \dv{t}\qty[\sum_i \ket{\psi_i}P_i\bra{\psi_i}]\\
        &=\sum_i \qty[\dv{t}(\ket{\psi_i})P_i\bra{\psi_i} + \ket{\psi_i}P_i\dv{t}(\bra{\psi_i})]\\
        &=\sum_i \qty[\frac{1}{i\hbar}\ket{\psi_i}P_i\bra{\psi_i} + \ket{\psi_i}P_i\frac{1}{-i\hbar}\bra{\psi_i}]\\
        &=\frac{1}{i\hbar}\vu{H}\vu{\rho}-\frac{1}{i\hbar}\vu{\rho}\vu{H}
    \end{align*}
    Entonces
    $$
    i\hbar \dv{t}\vu{\rho}=\comm{\vu{H}}{\vu{\rho}}
    $$   
\end{example}

\begin{example}[Evolución temporal de un valor medio]
    Sea un $A(t)$ un observable físico asociado al operador hermítico $\vu{A}(t)$. Considere cómo evoluciona con el tiempo el valor medio de $A(t)$, $\expval{A}$  dado el sistema $\ketpsi$.
    
    \textit{Solución}
    
    
    \begin{align*}
        \dv{t}\expval{A}&=  \dv{t}\ev{\vu{A}}{\psi} \\
         &=\dv{t}(\bra{\psi(t)})\vu{A}\ket{\psi(t)}+\qty[ \bra{\psi(t)}\dv{t}(\vu{A})\ket{\psi(t)}+\bra{\psi(t)}\vu{A}\dv{t}(\ket{\psi(t)})] \\
         &=\frac{1}{i\hbar}\ev{\vu{H}\vu{A}}{\psi}+\expval{\pdv{A}{t}}+\frac{1}{-i\hbar}\ev{\vu{A}\vu{H}}{\psi}\\
         &=\expval{\pdv{A}{t}}+\frac{i}{\hbar}\ev{\vu{H}\vu{A}-\vu{A}\vu{H}}{\psi}\\
         &=\expval{\pdv{A}{t}}+\frac{i}{\hbar}\ev**{\comm{\vu{H}}{\vu{A}}}{\psi}
    \end{align*}

\end{example}

\subsubsection{iii Postulado}

En el continuo del tiempo entre dos medidas consecutivas los estados puros de un sistema físico siguen siendo puros y cualquier vector de estado \ketpsi evoluciona de acuerdo a la ecuación de \perSch. ¿Un estado mezclado evoluciona en un estado mezclado?